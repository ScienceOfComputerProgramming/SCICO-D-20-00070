\documentclass{mproj}
\usepackage{graphicx}

\usepackage{url}
\usepackage{fancyvrb}
\usepackage[final]{pdfpages}
\usepackage{times}

% for alternative page numbering use the following package
% and see documentation for commands
%\usepackage{fancyheadings}


% other potentially useful packages
%\uspackage{amssymb,amsmath}
%\usepackage{url}
%\usepackage{fancyvrb}
%\usepackage[final]{pdfpages}

\begin{document}

%%%%%%%%%%%%%%%%%%%%%%%%%%%%%%%%%%%%%%%%%%%%%%%%%%%%%%%%%%%%%%%%%%%
\title{Source Control Integrated Issue Tracking}
\author{Nystrom Johann Edwards}
\date{7th September, 2018}
\maketitle
%%%%%%%%%%%%%%%%%%%%%%%%%%%%%%%%%%%%%%%%%%%%%%%%%%%%%%%%%%%%%%%%%%%

%%%%%%%%%%%%%%%%%%%%%%%%%%%%%%%%%%%%%%%%%%%%%%%%%%%%%%%%%%%%%%%%%%%
\begin{abstract}
There are many tools required for engineering large and complex software products. There are also many areas in which they overlap. This project investigates and proposes a tool as a solution for the overlapping qualities of source code version control systems and issue tracking in software project development. 

Present state of the art software project issue tracking tools, such as GitHub, GitLab, and JIRA store issues in a database alongside the version control repository that contains the project's source code. This creates friction in development efforts because software developers must remember to keep both the issue tracker and the version control repository up to date as progress is made on completing tasks.

In the next generation of issue tracking we utilise the power of the version control system to conduct issue tracking such that developers can work on one source of truth. Using a Source Control Integrated Issue Tracking system aims to eliminate friction by providing an extension to the version control system that manages issues embedded within the source code. Developers are now able to update issue metadata, track progress and inspect the relationships between their issues and source code.  %Issues co-exist with commits as objects that are tracked. 
\end{abstract}
%%%%%%%%%%%%%%%%%%%%%%%%%%%%%%%%%%%%%%%%%%%%%%%%%%%%%%%%%%%%%%%%%%%

%%%%%%%%%%%%%%%%%%%%%%%%%%%%%%%%%%%%%%%%%%%%%%%%%%%%%%%%%%%%%%%%%%%
\educationalconsent

%%%%%%%%%%%%%%%%%%%%%%%%%%%%%%%%%%%%%%%%%%%%%%%%%%%%%%%%%%%%%%%%%%%

\newpage
%%%%%%%%%%%%%%%%%%%%%%%%%%%%%%%%%%%%%%%%%%%%%%%%%%%%%%%%%%%%%%%%%%%
\section*{Acknowledgements}

I would like to thank my supervisor, Dr Timothy Storer for exposing me to new software engineering possibilities through this project, for his advice and mentoring on creating a new and useful software tool, and most importantly for assistance in making the difficult decisions on design choices.

I would like to thank my wife Virginia Jordan-Edwards for her support and encouragement throughout my programme and all her efforts to ensure that it was completed with great success. I would not have gotten this far without her valuable insight into my projects, her motivational speeches and her uplifting presence.

I am especially grateful to my sister Zophia who has financed my programme and has always believed in supported and encouraged my software development talent. She will always be one of my greatest supporters.

I am grateful to my mother Joslyn and all other members of my family that have invested in my education and encouraged me in this undertaking. There efforts will always be greatly appreciated.

Finally, I would like to thank my colleagues at the university that helped in the evaluation of the project that provided feedback needed to make a good solution.

%%%%%%%%%%%%%%%%%%%%%%%%%%%%%%%%%%%%%%%%%%%%%%%%%%%%%%%%%%%%%%%%%%%
\tableofcontents
%%%%%%%%%%%%%%%%%%%%%%%%%%%%%%%%%%%%%%%%%%%%%%%%%%%%%%%%%%%%%%%%%%%

%%%%%%%%%%%%%%%%%%%%%%%%%%%%%%%%%%%%%%%%%%%%%%%%%%%%%%%%%%%%%%%%%%%
\chapter{Introduction}\label{intro}

There are many tools used by software engineers in order to create high quality, high performance and maintainable software products. These tools are used by various levels within an organisation in order to coordinate efforts to this end. As such, they are exposed to a wide range of persons collaborating to meet this goal.

Issue tracking systems is at the fore front of collaborating tools when it comes to software development. Bertram argues that it is used by various groups of persons in order to have a common point of knowledge when it comes to the product that is being developed \cite{Bertram:2010}. For developers in particular the issue tracking system is used as a means to store information on development tasks. Tasks may include, the development of a new feature, the discovery of a bug, the preparation for a release, among others. When progress is made on these tasks it is recorded in the tracker and it provides key information to managers and the team what has taken place.

Version control systems by comparison allow for a similar type of collaboration, however at its core it is the working progress of the entire development of the software product. Here we see the overlap of the two systems where one is related to the product as a working growing artifact and the other as a planning and knowledge base on segments of the software product. Integrating the two systems would potentially provide the benefit of keeping track of both types of critical information in a format that does not duplicate work efforts.

We propose to introduce developers to an extension of the version control system that allows for the tracking of issues as a first step in testing this theory. It is expected that a tool such as this will allow developers to remain highly collaborative on issues and productive on writing code into version control thereby reducing the friction of maintaining two systems and switching focus during development.

%%%%%%%%%%%%%%%%%%%%%%%%%%%%%%%%%%%%%%%%%%%%%%%%%%%%%%%%%%%%%%%%%%%
\chapter{Analysis}\label{analysis}

%%%%%%%%%%%%%%%%%%%%%%%%%%%%%%%%%%%%%%%%%%%%%%%%%%%%%%%%%%%%%%%%%%%
\chapter{Design}\label{design}

\section{Implementation}


%%%%%%%%%%%%%%%%%%%%%%%%%%%%%%%%%%%%%%%%%%%%%%%%%%%%%%%%%%%%%%%%%%%
\chapter{Evaluation}\label{evaluation}

\section{Testing}


%%%%%%%%%%%%%%%%%%%%%%%%%%%%%%%%%%%%%%%%%%%%%%%%%%%%%%%%%%%%%%%%%%%
\chapter{Conclusion}\label{conclusion}

\section{Future Work}

\appendix % first appendix
%%%%%%%%%%%%%%%%%%%%%%%%%%%%%%%%%%%%%%%%%%%%%%%%%%%%%%%%%%%%%%%%%%%
\chapter{First appendix}

\section{Section of first appendix}

%%%%%%%%%%%%%%%%%%%%%%%%%%%%%%%%%%%%%%%%%%%%%%%%%%%%%%%%%%%%%%%%%%%
\chapter{Second appendix}

%%%%%%%%%%%%%%%%%%%%%%%%%%%%%%%%%%%%%%%%%%%%%%%%%%%%%%%%%%%%%%%%%%%
% it is fine to change the bibliography style if you want
\bibliographystyle{plain}
\bibliography{mproj}
\end{document}
